\documentclass[12pt,a4paper]{article}
\usepackage[utf8]{inputenc}
\usepackage[brazil]{babel}
\usepackage[T1]{fontenc}

\usepackage{times}

\title{Exercícios sobre Alguma Coisa}
\author{Universidade Estadual do Centro-Oeste (UNICENTRO)\\
        Departamento de Ciência da Computação (DECOMP)\\
        Matéria \\
        Professor X \\
        Você }

\newcommand{\pergunta}[1]{\vspace{10pt} \par \noindent \textbf{#1} \par}
\newcommand{\pitem}[1]{\vspace{5pt} \par \noindent \textbf{#1} \par}

\begin{document}

\maketitle

\pergunta{1. Pergunta um}

Resposta da pergunta 1.

%%%%%%%%%%%%%%%%%%%%%%%%%%%%%%%%%%%%%%%%%%%%%%%%%%%%%%%%%%%%%%%%%%%%%%%%%%%%%%%%
\pergunta{2. Esta é uma pergunta com itens:}

\pitem{a) Esse item exige uma resposta}
Clara e concisa.
\pitem{b) Já este, exige uma resposta}
Uma resposta, que eu diria que seguindo os padrões da não-padronização 
neo-globalizacional, é uma resposta não tende a se limitar uma gama curta
de palavras da língua portuguesa.  Assim, portanto, o exercício de ser um
falante nativo desta língua de guerra, diga-se de passagem, faz-se justíficavel.

\end{document}

