\documentclass[12pt,a4paper]{article}
\usepackage[utf8]{inputenc}
\usepackage[brazil]{babel}
\usepackage[T1]{fontenc}

\usepackage{times}

\usepackage{setspace}
\onehalfspacing

\hyphenation{o-pe-ra-ci-o-nal}
\hyphenation{o-pe-ra-ci-o-na-is}

\usepackage{fancyhdr}
\setlength{\headheight}{15.2pt}
\pagestyle{fancy}

%\usepackage{indentfirst}

%\setlength{\topmargin}{15mm}
\setlength{\topskip}{0.8in}
%\setlength{\textheight}{247mm}
%\setlength{\textwidth}{160mm}
%\setlength{\oddsidemargin}{15mm}
%\setlength{\evensidemargin}{10mm}

% recuo de 2,25 cm 
% espacamento 1 1/2
% fonte times new roman 	JUSTIFIED

\oddsidemargin 0.0in 

\title{Atividade Avaliativa 1}
\author{Paulo R. Urio \\
        Universidade Estadual do Centro-Oeste (UNICENTRO)\\
        Departamento de Ciência da Computação (DECOMP)\\
        Sistemas Operacionais I\\
        Professora: Evanise A. Caldas\\
        Acadêmico: Paulo R. Urio}
        
%\makeatletter

\begin{document}
\fancyhf{}
 
\lhead{Universidade Estadual do Centro-Oeste -- UNICENTRO \\
       Setor de Ciências Exatas e de Tecnologia -- SEET\\
       Departamento de Ciência da Computação -- DECOMP\\
       \textbf{Acadêmico:} Paulo Roberto Urio }
\rhead{Evanise Araujo Caldas \\
	Sistemas Operacionais I\\ 
	\today\\
	 \textbf{RA:} 570091403 }
\rfoot{\thepage}

\begin{enumerate}
\item
c) O Sistema Operacional funciona como uma interface entre o usuário e
computador.

\item
a) (\textbf{F}) A CPU ou UCP é um dispositivo lógico que controla o Sistema Operacional.\\
b) (\textbf{V}) Desfragmentador é um tipo de software utilitário.\\
c) (\textbf{V}) Dispositivos I/O são dispositivos de entrada e saída.\\
d) (\textbf{V}) A memória principal é uma unidade funcional do tipo volátil.\\

\item
a) (\textbf{F}) Um Sistema Operacional não permite as pessoas utilizarem o hardware do
computador.\\
b) (\textbf{V}) Sistemas Operacionais Monotarefas executam um aplicativo de cada vez.\\
c) (\textbf{F}) Nos Sistemas Operacionais Multitarefa o gerenciamento de memória é
simples.\\
d) (\textbf{V}) Os sistemas com múltiplos processadores caracteriza.

\item
Um computador sem um sistema operacional pode rodar programas desde que o 
programa esteja executando o \textsl{boot}. Um computador conhecido que
não possuía sistema operacional é o ENIAC, onde o programa era feito através
de 3.000 \textsl{switches} com conexões feitas por centenas de cabos.
Então um sistema operacional torna o computador mais fácil de usar, pois suas
duas principais funções são de controle de recursos e abstração de partes
complexas. O controle de recursos controla o tempo e espaço que cada programa
irá ter. E a abstração da complexidade é tornar o sistema mais fácil de 
ser programável. Assim o acesso aos recursos do sistema fica mais fácil e o
compartilhamento dos recursos se mantém organizado.

\item
A vantagem de sistemas que operam em rede é que há a possibilidade de 
compartilhamento. Por exemplo, cada sistema pode compartilhar suas impressoras
e seus dados com os outros sistemas da rede. Também os dados podem ser 
replicados por essa rede, gerando maior segurança dos casos caso algum nó
da rede falhe. Mas se não houver replicação de dados e um nó falhar, esta
pode ser uma desvantagem da rede.
Computadores \textsl{stand-alone} são computadores sem qualquer acesso a
rede local ou internet, de forma que todos os recursos estão nele. A 
desvantagem dele é a falta de segurança dos dados, não há replicação ou
compartilhamento.

\item
Máquina virtual é um ambiente, geralmente um \textsl{software} ou sistema 
operacional que não existe físicamente, mas é criado por um outro ambiente.
Neste contexto a VM (\textsl{Virtual Machine}, em inglês) é chamada de 
\textsl{guest} (convidado, em inglês) enquanto o ambiente que a criou é
chamado de \textsl{host}. Geralmente as VMs são criadas para
executar um conjunto de instruções diferente do que a do ambiente do 
\textsl{host}. O \textsl{host} controla os recursos para cada máquina virtual
criada, geralmente atribuíndo dinamicamente para cada uma. Um exemplo de
VM é a JVM (\textsl{Java Virtual Machine}) que interpreta comandos específicos
do Java. Ela executa um código chamado de \textsl{bytecode} abstraíndo os
recursos para este \textsl{bytecode}. A JVM não depende de conjunto de 
instruções ou das APIs específicas de cada sistema operacional. Ela virtualiza
todos os recursos para o \textsl{bytecode}.
O usuário que interage com um servidor virtualizado o vê como uma máquina
física, no sentido de que ele consegue acessar os recursos da máquina como
discos rígidos, memória RAM, processadores e conexões Ethernet. Na verdade
tudo isso é virtual. Por exemplo, ao invés de acessar o disco rígido real, o
usuário acessa um disco criado pelo ambiente \textsl{host}. Este disco que 
irá acessar o disco rígido real.
A vantagem disto é que as VMs podem ser ambientes diferentes do \textsl{host},
facilitando testes, aumentando a segurança, melhor uso do harware e também 
gera economia. Com é possível definir um próprio hardware para cada máquina
virtual, não é necessário gastar dinheiro em uma máquina física para cada
sistema.

\item
\begin{itemize}
\item \textbf{Sistema operacional de tempo-real} Sistema usado para controle de 
máquinas, instrumentos científicos e sistemas industriais. Possui uma
interface com o usuário bem pequena, pois não é um sistema flexível. A parte
mais importante é a de gerenciamento de recursos, todas as operações devem
ser executadas em um tempo pré-definido sempre que forem executadas. Se a 
operação for executada mais rápida ou mais lenta, será caracterizado como uma
falha.

\item \textbf{Single-user, single task} Sistema desenvolvido para que haja apenas um
usuário executando apenas uma tarefa por vez. Um exemplo é o Palm OS, sistema
para computadores embarcados.

\item \textbf{Single-user, multi-tasking} É o tipo de sistema mais comum para usuários
com laptops ou desktops. Microsoft Windows, Mac OS são exemplos de
plataformas que permitem um usuário executar mais de uma tarefa simultaneamente.

\item \textbf{Multi-user} Um sistema multi-usuário permite que vários usuários
tomem vantagem dos recursos do computador simultaneamente. O sistema deve se
certificar que a divisão de recursos para cada usuário esteja balanceada.
Unix, VMs e sistemas de mainframe são exemplos de sistemas operacionais de
multi-usuário.
\end{itemize}

\item
Ele tem a característica de não ter interação do usuário com o programa.
A entrada e saída de dados do programa é implementada para utilizar dados
em uma memória secundária, como arquivos em um disco rígido. Exemplos de
programas que fazem processamento batch são programas de cálculo numérico,
compiladores, programas de ordenação e backups. Nenhum destes exemplos exige
a interação do usuário.

\item
Sistemas de tempo compartilhado dão a liberdade de um programa executar por
intervalos pequenos logo deixando outro programa executar por um intervalo 
pequeno. Este método de executar um pouco de cada programa dá a sensação
ao usuário que o ambiente está executando todos os processos 
simultaneamente. Enquanto que no sistema de tempo compartilhado quanto mais
rápido executar melhor será a experiência do usuário, em sistemas de tempo
real o que importa é a pontualidade. Aplicações industriais e bélicas são
indicadas para sistemas de tempo real, como sistema de controle de produção
e robôs enviados para outros planetas.

\item
São sistemas que possuem processadores interligados trabalhando em conjunto.
A vantagem é poder dividir o mesmo programa entre os processadores ou dividir
cada processador para um programa.

\item
Em sistemas fortemente acoplados os recursos são compartilhados entre todos.
Nos fracamente acoplados não há compartilhamento e cada sistema precisa
fazer o controle independentemente dos outros sistemas.

\item
São sistemas interligados através de linhas de comunicação com cada um fazendo
de forma independente o gerenciamento de recursos como processadores, memória
e dispositivos de entrada e saída.

\item
Em sistemas operacionais de rede cada nó pode compartilhar seus recursos,
como impressores e arquivos com os outros nós da rede. Nos sistemas 
distribuídos, o sistema irá esconder detalhes dos nós individuais, fazendo
com que seja tratado como um conjunto único e um sistema fortemente acoplado.

\item
Processador, memória principal e dispositivos de entrada e saída.

\item
O processador é formado de uma Unidade de Controle (UC), uma Unidade Lógica e
Aritmética (ULA) e de Registradores. A UC gerencia as atividades dos 
componentes do computador, como leitura e escrita de dados. A ULA realiza
operações lógicas e aritméticas (comparações, somas e subtrações).

\item
A memória principal é em geral um dispositivo de armazenamento volátil, que
guarda dados e instruções utilizados pelo processador na execução de programas.
A memória secundária não é volátil, possui maior capacidade de armazenamento
mas é mais lenta para acessar os dados armazenados.

\item
Dispositivos de entrada e saída são dispositivos usados para entrada de dados
para um programa processar e saída de dados com os resultados do processamento.

\item
RISC possui um conjunto de instruções pequeno que dá facilidade para ser 
processado, pois o objetivo é que cada instrução seja executado em um ciclo 
de clock.
CISC tem um conjunto de instruções maior com a finalidade de facilitar a vida
do programador, com instruções que fazem o trabalho mais díficil. Essas
instruções tomam mais ciclos de clock para serem executados. Os programas
para CISCI são em geral bem menores, mas não mais rápidos que os RISC. 
Atualmente os processadores domésticos são desenvolvidos com o melhor de
cada arquitetura, criando uma arquitetura mista de RISC e CISC.


\end{enumerate}

\end{document}
